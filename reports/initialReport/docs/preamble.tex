\documentclass[11pt,a4paper]{article}
\usepackage[utf8]{inputenc}
\linespread{1.1} %%1.3 = one half spacing
%Table of Content depth
\setcounter{tocdepth}{3}
\setcounter{secnumdepth}{3}
%Page layout sizes
\usepackage[left=2cm,right=2cm,top=2cm,bottom=2cm,a4paper]{geometry}
\usepackage[T1]{fontenc}
\usepackage{amsmath}
\usepackage{amssymb}
\usepackage{booktabs}
\usepackage{enumitem}
	\setlist{nosep}
\usepackage{url}
\usepackage[numbers]{natbib}
\usepackage[nottoc,notlot,notlof,numbib]{tocbibind}
%\usepackage[bottom]{footmisc}
\usepackage[table,xcdraw,usenames,dvipsnames]{xcolor}
\usepackage[font=scriptsize]{caption}
	\captionsetup{labelfont=bf,textfont=bf}
	\DeclareCaptionType{equCaption}[][List of equations]
	\captionsetup[equCaption]{name=Equation}
\usepackage{subfig}
\usepackage{float}
\usepackage{wrapfig}
\usepackage{hyperref}
	\hypersetup{colorlinks=false,pdfborder=0 0 0}
\usepackage{cleveref}
	\crefname{equation}{equation}{equations}
	\crefname{figure}{figure}{figures}
	\crefname{table}{table}{tables}	
	\crefname{code}{code}{code}
\usepackage{makeidx}
\usepackage{graphicx}

\usepackage{mathtools}
	\DeclarePairedDelimiter\abs{\lvert}{\rvert}%

\usepackage{tikz}
	\def\checkmark{\tikz\fill[scale=0.4](0,.35) -- (.25,0) -- (1,.7) -- (.25,.15) -- cycle;} 
\usepackage{tikz-3dplot}
\usetikzlibrary{arrows,shapes,positioning,shadows,trees,calc}

\usepackage{lmodern}
\usepackage{kpfonts}
\usepackage{nth}
\usepackage{verbatim}

\definecolor{light-grey}{gray}{0.95}

\usepackage[]{algorithm2e}
\usepackage{listings} %for code
\lstset{ %
  backgroundcolor=\color{white},   % choose the background color; you must add \usepackage{color} or \usepackage{xcolor}
  basicstyle=\scriptsize,        % the size of the fonts that are used for the code
  breakatwhitespace=false,         % sets if automatic breaks should only happen at whitespace
  breaklines=true,                 % sets automatic line breaking
  commentstyle=\color{blue},   	   % comment style
  frame=single,                    % adds a frame around the code
  keepspaces=true,                 % keeps spaces in text, useful for keeping indentation of code (possibly needs columns=flexible)
  keywordstyle=\color{Violet},       % keyword style
  %numbers=left,                    % where to put the line-numbers; possible values are (none, left, right)
  %numbersep=5pt,                   % how far the line-numbers are from the code
  %numberstyle=\tiny\color{gray}, % the style that is used for the line-numbers
  %rulecolor=\color{black},         % if not set, the frame-color may be changed on line-breaks within not-black text (e.g. comments (green here))
  %stepnumber=1,                    % the step between two line-numbers. If it's 1, each line will be numbered
  tabsize=2,                       % sets default tabsize to 2 spaces
  title=\lstname                   % show the filename of files included with \lstinputlisting; also try caption instead of title
}

\setlength{\parindent}{1pt}
\setlength{\parskip}{1em}
\hyphenchar\font=-1
\sloppy