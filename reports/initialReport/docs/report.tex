%Max 1500w or 4 pages
%Deadline 17:00 on Tuesday 09-02-2016

\section{Introduction}

The project aims to create a micro-modelled road traffic simulation engine enabling inspection of various effects of changing key rules within the model. For team members, this project is centered around teamwork and also, as a less prominent addition, the learning and application of software engineering skills.

\section{Project Description} %~60% of the report
Describe your team aims for the projects. Outline what aims your team have set for the project and your strategy for
achieving those aims. A rough timetable may be appropriate and you may wish to break your aims into levels
 (e.g. mandatory / optional or we will first work on level 1 aims before moving to level 2 aims) to allow for
 unexpected issues that may arise. You must also describe how your initial progress developing your piece of software
 has gone and where you currently are relative to your aims.


\subsection{Project Outline}


research micro/macro
architecture
division into artifact (gui,simulator,log)




\subsubsection{Model}

As the micro-model implementation focuses on individual representation of cars as physical objects and individual tuning of their parameters the starting point chosen is based on the Nagel–Schreckenberg model. The roads in this model are represented as cellular automata\cite{Schreckenberg}. In this, each cell can either be occupied by an object or be empty. Cars can have different properties such as acceleration, speed, direction and can move to other free neighbouring cells. The model can be expanded through the use of graphs to connect roads.

In the early stages streets will be arranged to model simple road networks and, in time, more realistic ones.
Rules in the simulation will include driving rules such as crossings, lane changes, etc.., and laws of physics. That can be further expanded to include weather factors and engine emissions.

The simulation system will be used to gather relevant data for analysis that can include road network throughput, congestion effects, pollution and more generally reaction of the model to changing rules.

\subsubsection{Technical considerations} %%??


\subsubsection{}

	\begin{itemize}
		\item approach (conceptual and technical)
		\begin{itemize}
		    \item Base: build, test, IDE, rough architecture set up.


		\end{itemize}

		\item rough architecture
		\begin{itemize}
		    \item Component 1 - simulator
		    \item Component 2 - GUI/animation engine
		\end{itemize}

		\item technology choice: Java. Why?:
		\begin{itemize}
		    \item maximum intersection set of what team knows
		    \item many libraries and tools, incl. spatial, modelling, parallelism, etc.
		    \item it is cross-platform, and can run on all 3 major OSs (we have Windows and Mac in team)
		    \item Has GUI capabilities
		\end{itemize}

	\end{itemize}
\subsection{Timetable}


\begin{itemize}


	\item Rough plan (estimate). Splitting all 2 months into 4 rough periods:
	\begin{itemize}

	    \item By 9th of Feb - initial report submission time:
	    \begin{itemize}
	        \item Team structure ready
	        \item Team process setup (communication, teamwork, knowledge sharing, visibility)
	        \item Rough architecture outlined, some basics set for modelling
	    \end{itemize}

	    \item by March 1st - less than 1 month iteration
	    \begin{itemize}
	        \item Basic modelling ready: simple roads (multi-lane) with different physical (and mental) properties of objects
	        \item Basic tooling support: setting up testbed parameters, map, parameters of active objects
	        \item Basic measurements ready: log output, throughput measurements, ability to save sim runs for future inspection
	        \item Basic GUI ready: able to display animations either from live simulation or saved as a file
	    \end{itemize}

	    \item by March 25th - end of project development (+buffer). Could go in one or some of the directions:
	    \begin{itemize}
	        \item Larger size of the model: graph with many connections
	        \item Loading map data from actual OpenStreetMap sources (filtering highways, counting lanes, etc.)
	        \item Attempting in-model traffic lights and model parameters optimization with different algorithms
	        \item Something else here gentlemen?
	    \end{itemize}

	    \item by 1st of April - documentation and report
	    \begin{itemize}
	        \item Code prepared and ensured for correctness and visibility
	        \item Report ready and submitted.
	        \item Presentation ready, rehearsed
	    \end{itemize}

	\end{itemize}

	\item Things done to date
	\begin{itemize}
		\item (Should be by that moment) Basic simulation skeleton and architecture
		\item Graple deployment harness
		\item Basic log system
	\end{itemize}

\end{itemize}


\section{Project Organisation} %~40% of the report
The project is organized around the skills team member have, and tools which can be used to achieve best results. Main
principles are flat team organization (no management relations), peer interaction (reviews, help) and agile overall
approach to the development. The concept of project advancement is built on the notion of Minimal Viable Product and
iterative improvements with team-wide backlog prioretization.

Each team member bears equal rights and responsibilities, in addition to that project coordinator has an extra responsibility
of ensuring that the artifacts are delivered on time. Peer assesment is set as equal participation/equal assesment.

\subsection{Process and Teamwork}
\subsubsection{Cooperation}
%This is probably already covered in section 1 - the task%
The development process approach is revolving around main aims of the project - that being not only building the product
satisfying specification, but also learning form each other and attempting the agile and responsible teamwork model.

Tasks are distributed according to personal preferences and skills. Existing team member experience is used to the most
possible extent. Actual breakdown of the system to individual task happens in face-to-face meetings, where team members
can discuss actual tasks, their separation, interfaces, ask and provide aid if needed etc.

As soon as there are multiple opinions on any decision to be made, following procedure might be applied:

\begin{enumerate}
    \item Two conflicting parties try to persuade each other using logic
    \item In case consensus not found, all the team is invovled to the mutual persuasion
    \item In case consensus not yet found, voting is attempted (public, majority rule). Team size of 5 guarantees decision.
        Everyone should follow this decision even if they were opposing party.
\end{enumerate}

As each team member is anticipated to invests similar effort into the project, votes are always equal.
\subsubsection {Meetings}
There is a planned weekly iterated meeting on Tuesdays. Meeting agenda is created in advance, everyone can
contribute. Meeting minutes are added as comments to the appropriate Trello\footnote{Trello is an agile process and information organisation tool having card and board notion, moving the concept of sticky notes and board to software world https://trello.com/tour} card
. Meeting time is to be used mostly for synchronisation, mutual help and discussion of
 complex matters. Daily interaction should happen in on-line process using various tools.

\subsubsection{Development Process}
%Add a section task/module breakdown%
The team has decided to follow simplistic Kanban-like\cite{Ahmed} agile development process, with individual work items piled into backlog,
which is prioritised by all team members. Team members pick up the tasks from the top of the stack which match their skill set
and module specialisation. The team will adopt new techniques and elements of the process as project course demands.
The same agile continuous refinement and improvement process is applied to the architecture.

The selected process is development-centric, it defines rules of how features are specified, implemented, tested and eventually
merged into the main git repository, and how visibility is achieved. The card system allows for clear classification of an activity by specific team member, component, status, priority, and activity progress. Commenting is then used for information exchange on that specific activity. Activities can be features, bugs, non-code items, research items, etc. The board as a whole gives snapshot of the current project status, as well as providing full change history.

Unit testing has a dedicated place within the process. Every commit should be accompanied by appropriate unit tests, and code should have
inline documentation supplied. Obligatory post-commit peer code review aims at improving code quality as well as ensuring the rules are abided.

\subsubsection{Tools}

Due to distributed and asynchronous project nature, software tools play an important role in ensuring the overall success. Following are description of \emph{development tools} chosen for the project:

\begin{itemize}
	\item Git and GitHub. Git is used as a primary code exchange point, communication tool by using appropriate commit messages. Frequent updates are an
	important project practice. At the time of writing this report, team does not use forked feature branches and pull requests approach. It is anticipated that team will
	switch to this approach of using Git after delivery of Milestone~1:~Minimal Viable Product. \\GitHub is a primary tool for code review, as it allows for
	code commenting.
	\item Build System to ensure build portability and easy project model exchange.
	 Gradle\footnote{Gradle is a task-based sophisticated build automation tool for Java and more. http://gradle.org} has been chosen as a primary build tool for the project, since it allows for easy plugin architecture, is quick and easy to setup and IntelliJ IDEA has
	 great Gradle integration features. Gradle is used to automate routine jobs of build, test, assembly  and documentation generation,
	as well as provide more in-depths analysis tools such as providing test coverage metrics.\\Gradle wrapper is also committed to the main code repository, so the build
	is configuration-agnostic and requires Java Development Kit only in minimal setup.
	\item IntelliJ IDEA\footnote{We decided to use IntelliJ IDEA Community Edition, https://www.jetbrains.com/idea/download/}. Selected as an IDE of choice as it may greatly improve productivity and has means for extensive refactoring functionality. IDEA has an outstanding
	integration with Gradle builds, allowing to run the tasks from within.
	\item Unit Testing. jUnit 4, Mockito and Fest-assert\footnote{Following article has been used as a base for testing environment setup http://blog.codeleak.pl/2013/07/test-code-readability-improved-junit.html} libraries are used to provide reliability and regression-resistant code base. Unit testing is one of the key development
	activities in project scope.
\end{itemize}

In addition to development tools, following communication and documentation tools are of utter importance for the project to succeed:

\begin{itemize}
	\item Trello is a web- and mobile-based collaboration tool organising projects into the boards. With this distinctive feature, it is extremely easy to use Trello as a
	Kanban board with tasks represented with Trello cards, and a full visibility of the scope and current progress. In addition to storing tasks as cards on board, same cards
	may be used for convenient information storage and exchange, forming essentially the loosely coupled knowledge base of links, ideas and preliminary research
	results which may be easily searched upon by any team member.
	\item HipChat\footnote{Atlassian web chat product, based on open Jabber protocol, https://www.hipchat.com} is used as a primary communication tool. Being a webchat with separate rooms and offering web, mobile and desktop clients, it offers myriad of different integrations. For now, only the simple GitHub and Trello integrations are used. GitHub issues messages into development chat room when there is an activity (commits, pull requests) happening in the main code repository, while Trello notifies of card changes and comments. 
	\item PlantUML for diagrams. Since PlantUML allows to create diagrams using text files, they are extremely easy to version under same source code control as the main code. Approach of non-WYSIWYG editor for diagram also saves time on object arrangement greatly, forcing user to spend more time on semantics rather than positioning.
	\item LaTeX is used for generating project preliminary and final reports.
\end{itemize}

