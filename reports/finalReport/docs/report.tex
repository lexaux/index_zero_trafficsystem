%%Maximum length = smallestOf( 12,000 words, 35 pages )
%%Report file name must be "team_X.pdf" (where X is the team name)

%Your initial report set out your aims for the project which you may wish to modify in light of feedback received. The final report should detail out what you set to achieve, what you did achieve, how you achieved it, and your evaluation of your work. You are free to structure your report as you see fit, and different projects may naturally lead to different structures. An example structure for your report follows. Bear in mind that the main aim of your report is to show the examiners that you have done quality work: focus on the noteworthy, not the mundane; explain what the examiners cannot know rather than the obvious; and show that you understand your project’s weaknesses as well as its strengths.

\section{Introduction}
%Describe the context for the work and the problem you are addressing. Briefly summarise what you achieved in the project.

\section{Review}
%Describe related work.

\section{Requirements and Design}
%Describe the requirements you set for your project at the beginning and the design you have taken for your project. Focus on why you decided to tackle the problem in the way you did, and what effects that had on the design. You may also wish to mention the impact of team-working on your requirements and design.

\section{Implementation} 
%Describe the most significant implementation details, focussing on those where unusual or detailed solutions were required. Quote code fragments where necessary, but remember that the full source code will be included as an appendix. Explain how you tested your software (e.g. unit testing) and the extent to which you tested it. If relevant to your project, explain performance issues and how you tackled them.

\section{Team Work}
%Describe how you worked together, including the tools and processes you used to facilitate group work.

\section{Evaluation}
%Critically evaluate your project: what worked well, and what didn’t? How did you do relative to your plan? what changes were the result of improved thinking and what changes were forced upon you? how did your team work together? etc. Note that you need to show that you understand the weaknesses in your work as well as its strengths. You may wish to identify relevant future work that could be done on your project.

\section{Peer Assessment}
%In a simple table, allocate the 100 ‘points’ you are given to each team member. Valid values range from 0 to 100 inclusive. You may assign decimal values, but the entire points must add up to precisely 100.
