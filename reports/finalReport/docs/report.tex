%%Maximum length = smallestOf( 12,000 words, 35 pages )
%%Report file name must be "team_X.pdf" (where X is the team name)

%Your initial report set out your aims for the project which you may wish to modify in light of feedback received. The final report should detail out what you set to achieve, what you did achieve, how you achieved it, and your evaluation of your work. You are free to structure your report as you see fit, and different projects may naturally lead to different structures. An example structure for your report follows. Bear in mind that the main aim of your report is to show the examiners that you have done quality work: focus on the noteworthy, not the mundane; explain what the examiners cannot know rather than the obvious; and show that you understand your project’s weaknesses as well as its strengths.

\section{Introduction}
%Describe the context for the work and the problem you are addressing. Briefly summarise what you achieved in the project.

\section{Review}
%Describe related work.

\begin{itemize}
    \item SUMO
\end{itemize}

\section{Requirements and Design}
%Describe the requirements you set for your project at the beginning and the design you have taken for your project. Focus on why you decided to tackle the problem in the way you did, and what effects that had on the design. You may also wish to mention the impact of team-working on your requirements and design.

\section{Implementation} 
%Describe the most significant implementation details, focussing on those where unusual or detailed solutions were required. Quote code fragments where necessary, but remember that the full source code will be included as an appendix. Explain how you tested your software (e.g. unit testing) and the extent to which you tested it. If relevant to your project, explain performance issues and how you tackled them.
\begin{itemize}
    \item Log module
    \begin{itemize}
        \item explain why and usage scenarios
        \item add basic arch. uml here
        \item Log config option injection
        \item Output formats
    \end{itemize}
\end{itemize}


\begin{itemize}
  \item serialization
  \item desirealization
  \item add screenshots from trello \ldots
\end{itemize}

\section{Team Work}
%Describe how you worked together, including the tools and processes you used to facilitate group work.

\section{Evaluation}
%Critically evaluate your project: what worked well, and what didn’t? How did you do relative to your plan? what changes were the result of improved thinking and what changes were forced upon you? how did your team work together? etc. Note that you need to show that you understand the weaknesses in your work as well as its strengths. You may wish to identify relevant future work that could be done on your project.

\subsection{Performance}

Alex: probably worth making these numbers up to a nice table, wdyt?

Test machine: OS X, Intel Core I7-4770HQ, 2.2 GHz, 16GB RAM, SSD

Opening sample Paris, Arc de Triomphe
Loaded 284 road and 168 junction descriptions in 118 millis
Constructed graph of 1145 mapFeatures, 57 mapLinks, 35 traffic generators in 68 millis
Drawn all static objects (no debug details) in 70 millis
Drawn dynamic objects in 0 millis

Opening sample Manhattan/Battery Park
Loaded 212 road and 148 junction descriptions in 11 millis
Constructed graph of 829 mapFeatures, 81 mapLinks, 40 traffic generators in 5 millis
Drawn all static objects (no debug details) in 29 millis
Drawn dynamic objects in 0 millis

Opening sample Elephant and Castle strange roundabout
Loaded 101 road and 80 junction descriptions in 37 millis
Constructed graph of 454 mapFeatures, 18 mapLinks, 12 traffic generators in 4 millis
Drawn all static objects (no debug details) in 8 millis
Drawn dynamic objects in 0 millis


Opening sample Strand area
Loaded 781 road and 530 junction descriptions in 62 millis
Constructed graph of 2998 mapFeatures, 196 mapLinks, 105 traffic generators in 19 millis
Drawn all static objects (no debug details) in 41 millis
Drawn dynamic objects in 0 millis


Opening sample Straight road with 6 lanes
Loaded 1 road and 0 junction descriptions in 4 millis
Constructed graph of 9 mapFeatures, 12 mapLinks, 2 traffic generators in 0 millis
Drawn all static objects (no debug details) in 2 millis
Drawn dynamic objects in 0 millis

Opening sample Simple one-way square
Loaded 1 road and 1 junction descriptions in 3 millis
Constructed graph of 6 mapFeatures, 2 mapLinks, 2 traffic generators in 0 millis
Drawn all static objects (no debug details) in 1 millis
Drawn dynamic objects in 0 millis

Opening sample Buckingham Palace area
Loaded 275 road and 176 junction descriptions in 28 millis
Constructed graph of 1032 mapFeatures, 68 mapLinks, 41 traffic generators in 8 millis
Drawn all static objects (no debug details) in 21 millis
Drawn dynamic objects in 0 millis

Opening sample Whole Manhattan
Loaded 14763 road and 10429 junction descriptions in 13607 millis
Constructed graph of 58394 mapFeatures, 1055 mapLinks, 596 traffic generators in 468 millis
Drawn all static objects (no debug details) in 531 millis
Drawn dynamic objects in 0 millis

Opening sample Whole Greater London
Loaded 31101 road and 24314 junction descriptions in 32911 millis
Constructed graph of 120672 mapFeatures, 547 mapLinks, 391 traffic generators in 644 millis
Drawn all static objects (no debug details) in 1087 millis


\begin{itemize}
    \item Map construction
    \begin{itemize}
        \item OSM xml import (lane directions and such)
        \itme mention specifically the issue with the loosely coupled data/inaccurate data (human error, multiple representations etc.). We had to 'cut edges' to get it there.
        \item SUMO has that capability but implements its own tags and does some guess work
    \end{itemize}

    \item Log module
    \begin{itemize}
        \item coloured terminal window if more time,
        \item >easier to read
        \item >segregated from the console
        \item Custom Filters
        \item >filtering based on class name (origin)
        \item >filtering based on selection of multiple non-sequential log levels instead of everything below the set global level (granular control)
    \end{itemize}
\end{itemize}



\section{Peer Assessment}
%In a simple table, allocate the 100 ‘points’ you are given to each team member. Valid values range from 0 to 100 inclusive. You may assign decimal values, but the entire points must add up to precisely 100.
